\documentclass[a4paper, 11pt]{article}
\usepackage[utf8]{inputenc}
\usepackage[T1]{fontenc}
\usepackage[french]{babel}

\usepackage{lmodern}
\usepackage{textcomp}
\usepackage{ifthen} \usepackage{amsmath} \usepackage{amsfonts} \usepackage{amssymb} \usepackage{graphicx}
\usepackage{enumitem}
\usepackage{multicol}
\usepackage[titlepage]{polytechnique}
\usepackage[pdftex=true,
			colorlinks=true,
			linkcolor=black,
			filecolor=red,
			urlcolor=blue,
			bookmarks=true,
			bookmarksopen=true]{hyperref}


\title{Réunions PSC 2015-2016}
\author{\begin{minipage}{0.42\textwidth}
		\begin{flushright}
		(Coordinateur) Frank \textsc{Nielsen}
		\end{flushright}
		\end{minipage}
		\quad --- \quad
		\begin{minipage}{0.42\textwidth}
		\begin{flushleft}
		Thomas \textsc{Clausen} (Tuteur)
		\end{flushleft}
		\end{minipage} \\[2mm]
		Groupe INF07 \\
		\begin{minipage}{0.45\textwidth}
		\begin{center}
		\begin{tabular}{rl}
		Pierrick & \textsc{Allègre} \\
		Gustavo & \textsc{Castro} \\
		Clément & \textsc{Durand} \\
		Felipe & \textsc{Garcia}
		\end{tabular}
		\end{center}
		\end{minipage}
		\begin{minipage}{0.45\textwidth}
		\begin{center}
		\begin{tabular}{rl}
		Alexandre & \textsc{Harry} \\
		Francisco & \textsc{Ribeiro} \\
		Pierre-Alexandre & \textsc{Thomas} \\
		Guillaume & \textsc{Vizier}
		\end{tabular}
		\end{center}
		\end{minipage}}
\subtitle{Comptes-rendus}
\date{\today}

\begin{document}
%\pagestyle{} %ou plain, headings, empty
\maketitle
\tableofcontents\clearpage

\section{Lundi 31 août 2015}

\subsection{Organisation du PSC}
Globalement, il a été envisagé d'organiser le PSC en deux grandes étapes.
\subsubsection{Première étape : hacking}
Il s'agit de créer une application\footnote{Ou à défaut un dispositif/objet connecté, mais ce n'est pas souhaitable.} destinée à récupérer les données et échanges d'un autre téléphone, le tout \emph{sans contact physique}.

\begin{itemize}
	\item Trouver le matériel et les logiciels nécessaires\footnote{Limiter la quantité de matériel pour la portabilité.} pour hacker.
	\item Se renseigner sur le fonctionnement des applications « sécurisées » et leurs éventuelles failles.
	\item[Objectif :] Intercepter des données \emph{interprétables}.
\end{itemize}

Pour aller plus loin :
\begin{itemize}
	\item Faire une application fonctionnant pour différents modes du téléphone\footnote{Mode avion, Wi-Fi uniquement, Bluetooth, etc.}.
	\item Envoyer des données à un portable pour les faire exécuter (et donc pouvoir faire n'importe quoi avec un téléphone qui n'est pas suffisamment sécurisé).
\end{itemize}

\subsubsection{Seconde étape : protection}
Le but est de protéger le dispositif contre ce type d'attaque\footnote{Et non contre le décryptage. On considère ici qu'une donnée cryptée, c'est une donnée sécurisée \emph{tant qu'on n'a pas intercepté notre clé de cryptage}.}

Cette partie peut s'approcher en deux étapes\footnote{La première étant plus courte que la seconde, et moins fastidieuse.} :
\begin{enumerate}
	\item Réaliser une application sécurisée qui remplacerait une application non vraiment sécurisée. Par exemple, \emph{Telegram} a le même rôle que \emph{Viber}, mais \emph{a priori} les échanges y sont sécurisés.
	\item Réaliser une « application\footnote{Application n'est pas le terme exact, mais il s'agit dans tous les cas d'un bout de code qui est exécuté sur le téléphone.} » qui a pour rôle de sécuriser les autres applications, c'est-à-dire :
	\begin{itemize}
		\item De les protéger de l'extérieur (attaques similaires à la notre).
		\item De les protéger des autres applications (attaques utilisant une application virus).
	\end{itemize}
\end{enumerate} 

\subsubsection{Situation dans le temps}

L'objectif est d'avoir \emph{au moins} terminé la partie hacking d'ici la soutenance à mi-parcours. Dans l'idéal, avoir avancé\footnote{Ou même terminé} dans la réalisation de l'application sécurisée serait intéressant.

\subsubsection{Domaine de travail}

A priori, nous travaillerons essentiellement avec android.

\subsection{Devoirs}

\subsubsection{Mail à notre tuteur}
Un mail a été envoyé à Thomas \textsc{Clausen} pour :
\begin{itemize}
	\item Obtenir un rendez-vous pour le Lundi 14 septembre.
	\item Lui communiquer nos objectifs.
	\item Obtenir son avis.
	\item Avoir des informations sur les sources de renseignements possibles et utiles, et sur les étapes à suivre pour progresser correctement.
	\item Lui transmettre la liste des membres de notre groupe.
\end{itemize}

\subsubsection{S'instruire}

Pour commencer à programmer pour android : \href{https://openclassrooms.com/courses/creez-des-applications-pour-android/votre-premiere-application-1}{OpenClassrooms} et \href{http://www.tutos-android.com/introduction-programmation-android}{Tutos Android}.

\subsubsection{Se préparer}

Installer les outils utiles :
\begin{itemize}
	\item Eclipse
	\item AndroidStudio
\end{itemize}

Pour la configuration proxy : il y a le site des \href{https://moodle.polytechnique.fr/mod/page/view.php?id=319}{moodles}.

\subsubsection{Miscellaneous}
Autres choses envisageables pour se mettre dans le bain :
\begin{itemize}
	\item Se renseigner sur le fonctionnement de l'application \emph{Telegram}.
	\item Regarder la série \emph{Mister Robot}.
\end{itemize}

\subsection{Échéances}

\begin{description}
	\item[25 septembre] Proposition détaillée :
	\begin{itemize}
		\item Enjeu et motivation, objectif final
		\item État de l'art (approches concurrentes/alternatives)
		\item Étapes intermédiaires, échéancier
		\item Méthodes, organisation, répartition
		\item Moyens (mobilisables à l'école), achats à prévoir, etc.
		\item Partenaires (internes comme externes)
		\item Éventuels résultats préliminaires
		\item Références bibliographiques
	\end{itemize}
	\item[20 novembre] (\emph{au plus tard}) Réunion de lancement
	\begin{itemize}
		\item Présentation de la proposition détaillée
		\item Discussion
		\item Validation ou recadrage par l'encadrement
		\item Éventuelles co-supervisions
	\end{itemize}
	\item[15 janvier] Rapport intermédiaire :
	\begin{itemize}
		\item Résultats intermédiaires
		\item Point sur nos objectifs/réalisations
		\item Difficultés/opportunités rencontrées
		\item Éventuelle révision des objectifs et du programme
	\end{itemize}
	\item[22 avril] Rapport final (30 à 40 pages) et page publique
	\item[2-27 mai] Soutenance finale : 30 minutes de présentation, 20 de questions et discussions
\end{description}

\subsection{Organisation pratique}

Pour s'organiser : il y a le \href{https://annuel.framapad.org/p/PSC_X2014_CGAFPAFG}{Framapad}.

\section{Lundi 14 septembre 2015}

\subsection{Proposition détaillée}

Il faut produire un plan de la proposition détaillée, pour pouvoir répartir les tâches et avancer dans la rédaction.

Éléments nécessaires pour la proposition détaillée :
\begin{itemize}
	\item Enjeux et motivation, objectif final
	\item État de l'art, approches concurrentes ou alternatives
	\item Objectifs intermédiaires, échéancier
	\item Méthode d'organisation
	\item Identification des moyens (labos, ateliers, achats)
	\item Contribution de partenaires
	\item Éventuels résultats préliminaires
	\item Bibliographie
	\item 
\end{itemize}

Idée de plan :
\begin{enumerate}
	\item Introduction du sujet
		\begin{enumerate}
			\item Enjeux et motivation,
			\item État de l'art,
			\item Objectif final
		\end{enumerate}
	\item Organisation du projet
		\begin{enumerate}
			\item Méthode d'organisation
			\item Échéancier et objectifs intermédiaires
		\end{enumerate}
	\item Moyens à disposition et/ou nécessaires
		\begin{enumerate}
			\item Moyens (labos, ateliers, achats)
			\item Résultats préliminaires
			\item Bibliographie
		\end{enumerate}
\end{enumerate}

Répartition des tâches :
\begin{enumerate}
	\item Pierrick Allègre, Felipe Garcia, Alexandre Harry
	\item Francisco Ribeiro Eckhardt Serpa, Gustavo Castro, Clément Durand
	\item Guillaume Vizier, Pierre-Alexandre Thomas
\end{enumerate}

\section{Lundi 5 octobre 2015}

Séance sur les protocoles. Inscription sur GitHub. La semaine prochaine nous travaillerons sur l'utilisation de git et l'émission/réception de paquets.

\section{Lundi 12 octobre 2015}

Séance sur l'envoi de paquets : bilan de ce que nous avions fait dans la semaine précédente, essentiellement avec scapy.

Formation Git (par Clément D) : présentation de \verb!git! et de son utilisation, pour l'utiliser en situation de développement et de rédaction de rapports.

\clearpage
\section{Lundi 19 octobre 2015}

Présents : Guillaume Vizier, Alexandre Harry, Clément Durand.

Préparation de la réunion de lancement qui aura lieu le mardi 20 octobre 2015.

\begin{enumerate}
	\item Présentation du sujet
	\begin{enumerate}
		\item Enjeux et motivations inchangés
		\item Analyse de l'état de l'art \\
		Déboucher sur notre étude des points techniques.
		\item Objectifs \\
		Révision des objectifs à la baisse : connaissances peu accessibles, etc.
	\end{enumerate}
	\item Organisation du projet
	\begin{enumerate}
		\item Méthode d'organisation \\
		Division du travail, rôles au sein de l'équipe, mentionner la comm' externe et le besoin d'avoir une direction plus précise pour entrer en contact avec l'extérieur. Par où commencer ? Insister \\
		Rendez-vous \\
		Communication : mentionner la formation git \\
		Mentionner la possibilité de tenir les cadres informés et savoir à quel point et avec quelle précision ils veulent être informés.
		\item Échéances \\
		Objectifs intermédiaires : mentionner l'étude de protocoles, etc. permettant de lancer le groupe. \\
		Premier objectif concret : avoir un sniffer opérationnel fin novembre (capable de filtrer des paquets pour réagir à un type précis de paquets)
	\end{enumerate}
	\item Moyens et résultats
	\begin{enumerate}
		\item Moyens nécessaires \\
		Insister sur le besoin d'une route à suivre, de sources, en raison du manque de vision globale sur le sujet.
		\item Résultats préliminaires \\
		Nous avons une vision légèrement meilleure du sujet grâce au travail fourni depuis le début, qui nous a permis de faire connaissance avec les réseaux et certains outils (dig, scapy, traceroute, wireshark, ettercap, etc.) \\
		Réfléchir à la position relative de Wireshark et de la loi française/internationale.
	\end{enumerate}
\end{enumerate}

\paragraph{Attribution des rôles. } Alexandre \textsc{Harry} se chargera de la présentation du sujet, en prenant garde de ne pas développer sur les points techniques et de bien phraser à propos de revoir les objectifs à la baisse. Clément \textsc{Durand} se chargera de la partie organisation et échéances. Dans cette partie, Felipe \textsc{Garcia} sera chargé d'expliquer rapidement quels points techniques nous avons travaillé, en racontant la séance sur les protocoles sans forcément entrer dans les détails. Guillaume \textsc{Vizier} traitera la dernière partie en prenant garde de bien insister sur le besoin d'obtenir des informations du tuteur en terme de direction (manque de vision globale de notre part, etc.), et Clément \textsc{Durand} traitera de la légalité de ce que nous ferons cette année, afin de tenter de définir le cadre de travail.

\end{document}